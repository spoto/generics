\section{Related Work}\label{sec:related_work}

% \todo{Forse questa parte va spostata.}
% 
% 
% \textbf{Compilation of generics --}
% Generics is a way for writing code parametrized by type, in other words 
% a same piece of code can be written by using a type placehorder 
% (a \emph{type parameter}) and then instantiated for specific,
% concrete types. There exists two common ways to implement generics in 
% a programming language, that are often described in literature~\cite{generics_categories}
%  as \emph{heterogeneous} and \emph{homogeneous}. In the heterogeneous 
% approach the code is specialized for each instance
% of the generic parameter; this is the approach adopted by C++ \emph{templates}.
% Conversely, the homogeneous approach is the one provided by Java and .Net; 
% in this case, only one instance of the code is maintained and shared by 
% all generic instances.
% This implementation is based on the type \emph{erasure} mechanism, where the 
% generic parameter is replaced by the upwards bound of each instance, most 
% often \<Object>.
% Even if the heterogeneous approach is the safest, it is rarely applied in particular
% in resource-constrained applications, because the code size may dramatically increase
% as some code is duplicated~\cite{generics_embedded_systems}. For code in blockchain,
% the heterogeneous approach obliges one to reinstall all instantiations of the 
% generic code, with extra costs of gas.
% Conversely, the homogeneous approach ensures a smaller consumption
% of resources but can lead to the problems described in this paper. \\
% 
% \textbf{Smart contracts vulnerabilities --}

It has been estimated that, on average, software developers make from 100 to 150 errors
for every thousand lines of code~\cite{software_engineering}.
In 2002, the National Institute of Standards and Technology (NIST) estimates that the economic
costs of faulty software in the US is about tens of billions of dollars per year and represent
approximately just under 1 percent of the Nation's gross domestic product.
The effects induced by errors in software development are even worse when such pieces of software
are smart contracts. Indeed, it is usually impossible to change a smart contract once
it has been deployed, the immutability being one of its main characteristics, so that
errors are treated as intended behaviours. Moreover, smart contracts often store and
manage critical data such as money, digital assets and identities. For this reason, smart contracts
vulnerabilities and correctness are becoming important in literature\,\cite{smart_contracts_verification}. Possible solutions can be
classified into three main categories: (i) static analysis of EVM bytecode,
(ii) automatic rectification of EVM bytecode and (iii) development of new languages
for smart contracts.

Given the plurality of languages currently available for the design of smart contracts,
static analysis is usually performed directly on the Ethereum bytecode, in order to make
the solution general enough and promote its adoption. At this regards, SafeVM~\cite{safevm}
is a verification tool for Ethereum smart contracts that works on bytecode and exploits
the state-of-the-art verification engines already available for C programs.
The basic idea is to take as input a smart contract in compiled bytecode, that
can possibly contain some \<assert> or \<require> annotations, decompile it and convert it
into a C program with \<ERROR> annotations. This C program can be verified by using
existing verification tools.
%
In~\cite{hol_smart_constracts}, the authors propose a verification tool for Ethereum
smart contracts based on the use of the existing Isabelle/HOL tool, together with the
specification of a formal logic for Ethereum bytecode. More specifically, the desired properties
of the contracts are stated in pre/postcondition style, while the verification
is done by recursively structuring contracts as a set of basic blocks
down to the level of instructions.
%
Another tool for the analysis of Ethereum bytecode is EthIR~\cite{ethir}. This open-source tool
allows the precise decompilation into a high-level, rule-based representation. Given such
representation, properties can be inferred through available state-of-the-art analysis tools
for high-level languages. More specifically, EthIR relies on an extension of Oyente,
a tool that generates code control-flow graphs in order to derive
a rule-based representation of the bytecode.

Relatively to the automatic certification of smart contracts,
Solythesis~\cite{solythesis_solidity_validation} is a compilation tool for smart contracts
that provides an expressive language for specifying desired safety invariants.
Given a smart contract and a set of user defined invariants,
it is able to produce a new enriched contract that will reject all transactions
violating the invariants.
%
Another solutions, based on bytecode rewriting, is presented in~\cite{bytecode_rewriting},
where the authors propose the enforcement of security policies through the enhancement of bytecode.
More specifically, the disassembled bytecode is instrumented through new security guard code
that enforces the desired policy. Their initial efforts are mainly focused on the verification
of arithmetic operations, such as the prevention of overflows. In the future, they plan to focus on
verifying memory access operations.
%
SMARTSHIELD~\cite{smartshield} is another tool for automatically
rectifying bytecode with the aim to fix three typical security bugs in smart contracts:
(i) state changes after external calls, (ii) missing checks for out-of-bound arithmetic operations,
and (iii) missing checks for failing external calls. More specifically, given an identified issue,
the tool performs a semantic-preserving code transformation to ensure that only the insecure code
patterns are revised, eventually sending the rectification suggestions back to the developers
when the eventual fixes can lead to side effects. The tool not only guarantees that the rectified
contracts are immune to certain attacks but also that they are gas-friendly.
Indeed, it adopts heuristics to optimize gas consumption. 

Given the plurality of verification tools that have been developed for indentifying security bugs
in smart contracts, a unified and standard approach for evaluating their accuracy and performances
have been developed in~\cite{effectiveness_analysis_tools}. This automated and systematic approach,
called SolidiFI, consists in injecting bugs into all potential locations in a smart contract,
in order to introduce different kinds of vulnerabilities, and then using several static analysis
tools to identify which bugs each of them is able to detect.

Finally, as regards to the definition of new programming languages for safe smart contracs,
Scilla~\cite{scilla} has been tailored by taking System~F as a foundational calculus.
It is able to provide strong safety guarantees by means of type soundness.
Thanks to its minimalistic nature, it has been possible to define also a generic and extensible
framework for lightweight verification of smart contracts by means of user-defined domain-specific
analyses. The type variables of the functional foundational calculus can be seen as
generic types. We do not know how they are compiled and if the strong typing guarantee of the
source code extends to the compiled code as well. Scilla contracts are developed with
the Neo Savant online IDE. Currently, neither Neo Savant IDE nor the block explorer
allow one to inspect the compiled bytecode.
