\section{Introduction}\label{sec:introduction}

\IEEEPARstart{B}{lockchains} exploit the redundant, concurrent execution of the same
transactions on a decentralized network of many machines,
in order to enforce their execution in accordance with
a set of predefined rules. Namely, blockchains make it hard, for a single machine,
to disrupt the semantics of the transactions or their ordering: a misbehaving single machine
gets immediately put out of consensus and isolated. Bitcoin~\cite{Nakamoto08,book-mastering-bitcoin}
has been the first blockchain's success story. Here
transactions are programmed in a non-Turing complete bytecode language,
almost exclusively used to implement transfers of units of coins between \emph{accounts}.

A few years after Bitcoin, another blockchain, called
Ethereum~\cite{Buterin13,AntonopoulosW18}, introduced the possibility of programming
transactions in an actual, imperative and Turing-complete programming language, called Solidity.
Solidity's code is organized in \emph{smart contracts}, that can be seen as
objects that control money. \todo{A smart contract is essentially an agreement between two or more parties that can be automatically enforced without the need for a trustworthy intermediary~\cite{ebp}.}
Ethereum's transactions can hence execute much more than coin transfers. Namely,
they run object constructors and methods, which results in a sort
of \emph{world computer} that persists the same objects inside the memory of all the
computers composing the blockchain's network.

In Solidity's bytecode,
non-primitive values are referenced through a very general
\<address> type. For instance, a Solidity method
\<child(Person p, uint256 n) returns Person> actually compiles
into \<child(address p, uint256 n) returns address>, losing most
type information~\cite{CrafaPZ19}.
Since, at run time, it is the bytecode that gets executed,
everything can be passed for \<p>, not just a \<Person> instance.
The compiler cannot even enforce strong typing
by generating defensive type instance checks and casts, because
values are unboxed in Ethereum: they have no attached
type information at run time,
they are just numerical \emph{addresses}.
It follows that \todo{inside the \<child> method, an eventual} call to a \<Person>'s method
on \<p> might actually execute any arbitrary code, if \<p> is not a \<Person>.
In other words, Solidity is not strongly typed.
Consequently, it is highly discouraged, in Solidity, to call methods on parameters passed
to another method, such as on \<p> passed to \<child>, since an attacker can pass crafted
objects for \<p>, with arbitrary implementations for their methods,
which can result in the unexpected execution of
dangerous code. This actually happened in the case of the infamous DAO hack~\cite{dao16}, that
costed millions of dollars.

Strong typing is one of the reasons that push towards the adoption
of \emph{traditional} programming languages for smart contracts. For instance,
the Cosmos blockchain~\cite{cosmos} uses Go. The
Hotmoka blockchain~\cite{hotmoka} uses a subset of Java
for smart contracts, called Takamaka~\cite{Spoto19,Spoto20}.
Hyperledger~\cite{hyperldeger} allows Go and Java.
Another reason is the availability of modern
language features, that are missed in Solidity,
such as \emph{generics}\ie the possibility of using
type variables. Generics are a powerful and very useful facility for programming
smart contracts, \todo{since they allow one to personalize the behaviour of such contracts and partially overcome the their inherent incompleteness\,\cite{ebp}}. In Java source code, generics are strongly typed, if no \emph{unchecked operations}
are used~\cite{NaftalinW06}, as it will always be the case in this paper.



There exists two common ways to implement generics in a programming language,
that are often described in literature~\cite{generics_categories} as \emph{heterogeneous}
and \emph{homogeneous}. In the heterogeneous approach the code is specialized for each instance
of the generic parameter; this is the approach adopted by C++ \emph{templates}.
Conversely, the homogeneous approach is the one provided by Java and .Net; in this case,
only one instance of the code is maintained and shared by all generic instances.
This implementation is based on the type \emph{erasure} mechanism, where the generic parameter
is replaced by the upwards bound of each instance, mostly often \<Object>.
Even if the heterogeneous approach is the safest, it is rarely applied in particular
in resource-constrained applications, because the code size may dramatically increase
as some code is duplicated~\cite{generics_embedded_systems}. For code in blockchain,
the heterogeneous approach obliges one to reinstall all instantiations of the generic code,
with extra costs of gas.
Conversely, the homogeneous approach ensures a smaller consumption
of resources but it can introduce the same kind of attack to the bytecode
as normal reference types do in Solidity, as we will describe in this paper.

The contribution of this paper is to show a real-life
use of generics for an actual smart contract contained in the support
library of the Takamaka language, and to demonstrate that a na\"{i}ve use
of Java generics can lead to a code security vulnerability which
allows an attacker to earn money by exploiting someone else's work, \todo{with both economical and legal side effects}.
This paper will provide a fix to that specific issue,
by \todo{proposing a re-engineering of the code that forces} the compiler to generate defensive checks.
More generally, this paper can be useful for the definition of
bytecode languages for future smart contract languages, by
learning from the weaknesses of Java bytecode.

The remainder of this paper is organized as follows.
Sec.~\ref{sec:shared_entities} introduces our real-life Java smart
contracts that uses generics. Sec.~\ref{sec:attack} shows that a na\"{i}ve
deployment of that contract leads to a code vulnerability.
Sec.~\ref{sec:fix} shows a fix to that vulnerability.
Sec.~\ref{sec:related_work} discusses some related work and, finally,
Sec.~\ref{sec:conclusion} concludes.

