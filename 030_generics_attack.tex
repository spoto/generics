\section{An Attack to the Shared Entities Contract}\label{sec:attack}

This paper originated from an actual security issue that we found in our code
that used shared entities in order to model the validators of a blockchain.
Namely, the Hotmoka blockchain is built over Tendermint~\cite{Kwon14}, a
generic engine for replicating an application over a network of nodes. In our case,
the application is the executor of smart contracts in Java, such as that in
Fig.~\ref{fig:simple_shared_entity}. Tendermint is based on a proof of stake
consensus, which means that a selected dynamic subset of the nodes is in charge of
validating the transactions and voting their acceptance. In general, validator nodes are
represented by \<Validator> objects, that are externally owned accounts
with an extra identifier. In the specific case of Tendermint,
validators are \<TendermintED25519Validator> objects whose
identifier is derived from their ed25519 public key (see Fig.~\ref{fig:validator}).
This identifier is public information, reported in the blocks or easily eavesdropped.
Tendermint applications can implement their own
policy for changing the validator set dynamically.
The \<AbstractValidators> class implements the validator set
and the distribution of the reward and is a subclass of \<SimpleSharedEntity>
(see Fig.~\ref{fig:abstract_validators}).
Shares are voting power in this case.
Its subclass \<TendermintValidators> restricts the type \<S> of the shareholders
to \<TendermintED25519Validator>.
At each block committed, Hotmoka calls the \<reward> method of \<AbstractValidators>
in order to reward the validators that behaved correctly
and punish those that misbehaved. They are specified by two strings
that contain the identifiers of the validators, as provided by the underlying
Tendermint engine.

\begin{figure}[t]
  \begin{center}
    \begin{lstlisting}[language=Takamaka]
public abstract class Validator 
       extends ExternallyOwnedAccount {

  public Validator(String publicKey) {
    super(publicKey);
  }

  @View
  public abstract String id();
}

public final class TendermintED25519Validator 
       extends Validator {

  // the first 40 characters of the hexadecimal representation
  // of the sha256 hashing of the public key bytes
  private final String id;

  public TendermintED25519Validator(String publicKey) {
    super(publicKey);
    this.id = /* derived from publicKey as in the comment for id above */
  }

  @Override @View
  public final String id() {
    return id;
  }
}
    \end{lstlisting}
  \end{center}
  \caption{The representation of a validator of a Tendermint blockchain.
  Its full code is available at \url{https://github.com/Hotmoka/hotmoka/blob/master/io-takamaka-code/src/main/java/io/takamaka/code/governance/tendermint/TendermintED25519Validator.java}.}\label{fig:validator}
\end{figure}

\begin{figure}[t]
  \begin{center}
    \begin{lstlisting}[language=Takamaka]
public abstract class 
    AbstractValidators<V extends Validator>
    extends SimpleSharedEntity<V, Offer<V>> {

  public AbstractValidators
    (V[] validators, BigInteger[] powers) {
    super(validators, powers);
  }

  public void reward
    (String behaving, String misbehaving, 
     BigInteger gasConsumed) { ... }
}

public class TendermintValidators
    extends AbstractValidators<TendermintED25519Validator> {

  public TendermintValidators
        (TendermintED25519validator[] validators, 
         BigInteger[] powers) {
    super(validators, powers);
  }
}
    \end{lstlisting}
  \end{center}
  \caption{The shared entity of the validators of a Hotmoka blockchain.
  Its full code is available at \url{https://github.com/Hotmoka/hotmoka/blob/master/io-takamaka-code/src/main/java/io/takamaka/code/governance/AbstractValidators.java}.}\label{fig:abstract_validators}
\end{figure}

Since \<SimpleSharedEntity> allows shares to be sold and bought, this holds for
its \<TendermintValidators> subclass as well: the set of validators
is dynamic and it is possible to sell and buy voting power in order to invest in the blockchain
and earn rewards at each block committed. At block creation time,
Hotmoka calls method \<getShareholders> inherited from
\<SimpleSharedEntity> and informs the
underlying Tendermint engine about the identifiers of the validator nodes for the next blocks.
Tendermint expects such validators to mine and vote the subsequent blocks, until a change in the
validators set occurs.

It is important that the \emph{Consistency of Shareholders} property holds
for \<TendermintValidators>: its shareholders
must be \<TendermintED25519Validator>s (as declared in the generic
signature of \<TendermintValidators>) that enforce a
match between their public key, that identifies who can spend the rewards sent to the validator,
and their Tendermint identifier, that identifies which node of the blockchain
must do the validation work (Fig.~\ref{fig:validator}).
If it were possible to add a shareholder of another
type \<Attacker>, the latter could decouple the node identifier from its
public key (see Fig.~\ref{fig:attacker}):
Tendermint would expect the node (belonging to the \emph{victim}) to do
the validation work while the owner
of the private key of the \<Attacker> could just wait for accrued awards to spend.
A sort of validator's slavery.
Sec.~\ref{sec:shared_entities} asserted that the \emph{Consistency of Shareholders}
property holds, at source level.
Namely, an \<attacker> (of type \<Attacker>) can only become shareholder
by accepting an ongoing sale \<offer> of shares through a call to
\<tv.accept(offer.cost, attacker, offer)> (see Fig.~\ref{fig:simple_shared_entity}).
This is impossible at source level, where that call does \emph{not} compile, since \<attacker> has type \<Attacker> that
is not an instance of \<S>, which has been set to \<TendermintED25519Validator>.
But a Hotmoka blockchain contains only the bytecode of \<SimpleSharedEntity>,
where the signature of \<accept> has been erased into
\<accept(BigInteger amount, PayableContract buyer, Offer offer)>.
Hence a blockchain transaction that invokes \<tv.accept(offer.cost, attacker, offer)>
at bytecode level
\emph{does} succeed, since \<attacker> is an externally owned account and all such accounts
are instances of \<PayableContract>.
That transaction adds \<attacker> to the shareholders of \<tv>,
therefore violating the \emph{Consistency of Shareholders} property and allowing
validator's slavery.

\begin{figure}[t]
  \begin{center}
    \begin{lstlisting}[language=Takamaka]
public class Attacker extends ExternallyOwnedAccount {

  public Validator(String publicKey) {
    super(publicKey);
  }

  @View
  public String id() {
    return /* id of the victim blockchain node, unrelated to publicKey */
  }
}
    \end{lstlisting}
  \end{center}
  \caption{An attacker that exploits a blockchain node for validation and fraudolently earns the rewards of its work.}\label{fig:attacker}
\end{figure}

