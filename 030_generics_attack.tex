\section{An Attack to the Shared Entities Contract}\label{sec:attack}

This paper originated from an actual security issue that we found in our code
that used shared entities in order to model the validators of a blockchain.
Namely, the Hotmoka blockchain is built over Tendermint~\cite{Kwon14}, a
generic engine for replicating an application over a network of nodes. In our case,
the application is the executor of smart contracts in Java, such as that in
Fig.~\ref{fig:simple_shared_entity}. Tendermint is based on a proof of stake
consensus, which means that a selected dynamic subset of the nodes is in charge of
validating the transactions and voting their acceptance. Validator nodes are
represented by \<Validator> objects, that are externally owned accounts
with an extra identifier derived from their public key (see Fig.~\ref{fig:validator}).
This identifier is public information, reported in the blocks or easily eavesdropped.
At each block, the reward for the block is distributed to the validators.
Tendermint applications can implement their own
policy for changing the validator set dynamically. In our case, the validator set
and the distribution of the reward is implemented by a subclass
\<Validators> of \<SimpleSharedEntity> whose type \<S> for the shareholders
has been set to \<Validator>. Shares are voting power in this case.

\begin{figure}[t]
  \begin{center}
    \begin{lstlisting}[language=Takamaka]
public final class Validator extends ExternallyOwnedAccount {
  private final String id;

  public Validator(String publicKey) {
    super(publicKey);
    this.id = /* derived from publicKey as specified in Tendermint's spec */
  }

  public @View String id() {
    return id;
  }
}
    \end{lstlisting}
  \end{center}
  \caption{The representation of a validator of a Tendermint blockchain.
  Its full code is available at \url{https://github.com/Hotmoka/hotmoka/blob/master/io-takamaka-code/src/main/java/io/takamaka/code/governance/tendermint/TendermintED25519Validator.java}.}\label{fig:validator}
\end{figure}

Since shares of a \<SimpleSharedEntity> can be sold and bought, the set of validators
is dynamic. Users can sell or buy voting power in order to invest in the blockchain
and earn rewards. At each block creation, Hotmoka calls method \<getShareholders>
on the shared entity and informs the
underlying Tendermint engine about the identifiers of the validator nodes for the next blocks.
Tendermint expects such validators to mine and vote the subsequent blocks, until a change in the
validators set occurs.

It is important that the shareholders be instances of \<Validator>, since that class enforces the
match between their public key, that identifies who can use the reward sent to the validator,
and the Tendermint identifier of the validator, that identifies which node of the blockchain
must do the validation work (Fig.~\ref{fig:validator}).
If it were possible to add a shareholder of another
type \<Attacker>, the latter could declare the identifier of a node that does not correspond to its
public key (see Fig.~\ref{fig:attacker}):
the node (belonging to the \emph{victim}) would do the work while the owner
of the private key of the \<Attacker> could just wait and earn the reward.
A sort of validator's slavery.

\begin{figure}[t]
  \begin{center}
    \begin{lstlisting}[language=Takamaka]
public class Attacker extends ExternallyOwnedAccount {

  public Validator(String publicKey) {
    super(publicKey);
  }

  public @View String id() {
    return /* id of the victim node of the blockchain */
  }
}
    \end{lstlisting}
  \end{center}
  \caption{An attacker that exploits a blockchain node for validation and fraudolently earns the reward of its work.}\label{fig:attacker}
\end{figure}

