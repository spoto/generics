\documentclass{llncs}

\usepackage{url}
\usepackage{graphicx}
\usepackage[T1]{fontenc}
\usepackage[hidelinks]{hyperref}
\usepackage{pdfpages}
\usepackage{relsize}


% Solidity
\newcommand{\uint}{\texttt{uint}}
\newcommand{\SafeMath}{\texttt{SafeMath}}
\newcommand{\Storage}{\texttt{Storage}}
\newcommand{\Contract}{\texttt{Contract}}
% Java
\newcommand{\UBI}{\texttt{UnsignedBigInteger}}
\newcommand{\UBIshort}{\texttt{UBI}}
\newcommand{\BI}{\texttt{BigInteger}}
\newcommand{\ie}{\textit{, ie.\ }}

\def\codesize{\smaller}
\def\<#1>{\codeid{#1}}
\newcommand{\codeid}[1]{\ifmmode{\mbox{\codesize\ttfamily{#1}}}\else{\codesize\ttfamily #1}\fi}

\begin{document}

\title{Power and Pitfalls of Generic Smart Contracts}
\titlerunning{Power and Pitfalls of Generics for Smart Contracts}
\author{Andrea Benini \and Sara Migliorini \and Fausto Spoto}
\institute{Dipartimento di Informatica, Universit\`a di Verona, Italy}

\maketitle

\begin{abstract}
  TBD
  \keywords{smart contract \and generics \and blockchain}
\end{abstract}

\section{Introduction}\label{sec:introduction}

Blockchains exploit the redundant, concurrent execution of the same
transactions on a decentralized network of many machines
in order to enforce their execution in accordance to
a set of predefined rules. Namely, blockchains make it hard, for a single machine,
to disrupt the semantics of the transactions or their ordering: a misbehaving single machine
gets immediately put out of consensus and isolated. Bitcoin~\cite{Nakamoto08,book-mastering-bitcoin}
has been the first blockchain's success story. Here
transactions are programmed in a non-Turing complete bytecode language,
almost exclusively used to implement transfers of units of coins between \emph{accounts}.

A few years after Bitcoin, another blockchain, called
Ethereum~\cite{Buterin13,AntonopoulosW18}, introduced the possibility of programming
transactions in an actual, imperative and Turing-complete programming language, called Solidity.
Solidity's code is organized in \emph{smart contracts}, that can be seen as
objects that control money.
Ethereum's transactions can hence execute much more than coin transfers. Namely,
they run object constructors and methods, which results in a sort
of \emph{world computer} that persists the same objects in the memory of all the
computers in the blockchain's network.

In Solidity's bytecode,
non-primitive values are referenced through a very general
\<address> type. For instance, a Solidity method
\<child(Person p, uint256 n) returns Person> actually compiles
into \<child(address p, uint256 n) returns address>, losing most
type information~\cite{CrafaPZ19}.
At run time, it is the bytecode that gets executed. Hence,
everything can be passed for \<p>, not just a \<Person>.
The compiler cannot even enforce strong typing
by generating defensive type instance checks and casts, since
values are unboxed in Ethereum: they have no attached
type information at run time,
they are just numerical \emph{addresses}.
It follows that calls to \<Person>'s methods
on \<p> might actually execute arbitrary code, if \<p> is not a \<Person>.
In other words, Solidity is not strongly typed.
Consequently, it is highly discouraged, in Solidity, to call methods on parameters passed
to our methods, such as on \<p> passed to \<child>, since an attacker can pass crafted
objects for \<p>, with arbitrary implementations for their methods,
which can result in the unexpected execution of
dangerous code. This actually happened in the case of the infamous DAO hack~\cite{dao16}, that
costed millions of dollars.

Strong typing is one of the reasons that push towards the adoption
of \emph{traditional} programming languages for smart contracts. For instance,
the Cosmos blockchain~\cite{cosmos} uses Go. The
Hotmoka blockchain~\cite{hotmoka} uses a subset of Java
for smart contracts, called Takamaka~\cite{Spoto19,Spoto20}.
Hyperledger~\cite{hyperldeger} allows Go and Java.
Another reason is the availability of modern
language features, missing in Solidity,
such as \emph{generics}\ie the possibility of using
type variables. Generics are powerful and very useful for programming
smart contracts. In Java source code, they are strongly typed, if no \emph{unchecked operations}
are used~\cite{NaftalinW06}, as it will always be the case in this paper.
However, generics in Java are compiled by \emph{erasure}\ie replaced
by their upwards bound, most often \<Object>. Hence, the risk
is that generics introduce the same kind of attack to the bytecode
as normal reference types do in Solidity.

The contribution of this paper is to show a real-life
use of generics for an actual smart contract used in the support
library of the Takamaka language, and to show that a naive use
of Java generics can lead to a code security vulnerability that
allows an attacker to earn money by exploiting someone else's work.
This paper will provide a fix to that specific issue,
by obliging the compiler to generate defensive checks.
More generally, this paper can be useful for the definition of
bytecode languages for future smart contract languages, by
learning from the weaknesses of Java bytecode.

The rest of this paper is organized as follows.
Sec.~\ref{sec:shared_entities} introduces our real-life Java smart
contracts that uses generics. Sec.~\ref{sec:attack} shows that a naive
deployment of that contract leads to a code vulnerability.
Sec.~\ref{sec:fix} shows a fix to that vulnerability.
Sec.~\ref{sec:conclusion} concludes.

\section{Shared Entities using Generics}\label{sec:shared_entities}

\section{An Attack to the Shared Entities Contract}\label{sec:attack}

\section{Fixing the Compilation of the Shared Entities Contract} \label{sec:fix}

\section{Conclusion}\label{sec:conclusion} 

\bibliographystyle{plain}
\bibliography{biblio}

\end{document}
