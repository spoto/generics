\documentclass{llncs}

\usepackage{url}
\usepackage{graphicx}
\usepackage[T1]{fontenc}
\usepackage[hidelinks]{hyperref}
\usepackage{pdfpages}
\usepackage{relsize}


% Solidity
\newcommand{\uint}{\texttt{uint}}
\newcommand{\SafeMath}{\texttt{SafeMath}}
\newcommand{\Storage}{\texttt{Storage}}
\newcommand{\Contract}{\texttt{Contract}}
% Java
\newcommand{\UBI}{\texttt{UnsignedBigInteger}}
\newcommand{\UBIshort}{\texttt{UBI}}
\newcommand{\BI}{\texttt{BigInteger}}
\newcommand{\ie}{\textit{, ie.\ }}

\def\codesize{\smaller}
\def\<#1>{\codeid{#1}}
\newcommand{\codeid}[1]{\ifmmode{\mbox{\codesize\ttfamily{#1}}}\else{\codesize\ttfamily #1}\fi}

\begin{document}

\title{Power and Pitfalls of Generic Smart Contract}
\titlerunning{Power and Pitfalls of Generics for Smart Contracts}
\author{Andrea Benini \and Sara Migliorini \and Fausto Spoto}
\institute{Dipartimento di Informatica, Universit\`a di Verona, Italy}

\maketitle

\begin{abstract}
  TBD
  \keywords{smart contract \and generics \and blockchain}
\end{abstract}

\section{Introduction}\label{sec:introduction}

Blockchains exploit the redundant, concurrent execution of the same
transactions on a decentralized network of many machines
in order to enforce their execution in accordance to
a set of predefined rules. Namely, blockchains make it hard, for a single machine,
to disrupt the semantics of the transactions or their ordering: a misbehaving single machine
gets immediately put out of consensus and isolated. Bitcoin~\cite{Nakamoto08,book-mastering-bitcoin}
has been the first blockchain's success story. Here
transactions are transfers of units of coins between accounts, with the specific rule that
the same inputs cannot be spent twice. Bitcoin's coins are called bitcoins
themselves and have been the first example of a blockchain \emph{token}.

A few years after Bitcoin, another blockchain, called
Ethereum~\cite{Buterin13,AntonopoulosW18}, introduced the possibility of programming
transactions with an actual, imperative programming language, called Solidity.
Ethereum's transactions are still paid in terms of its native \emph{ether} token,
but they execute much more than token transfers. Namely,
transactions run object constructors and methods, which results in a sort
of \emph{world computer} that persists the same objects in the memory of all the
computers in the blockchain's network. These computers must execute
the transactions according to Solidity programs called
\emph{smart contracts}, or will end up being put out of consensus. Among such programs,
a few have emerged, that implement a dynamic ledger of coin transfers between accounts. These
coins are not the native ether, but rather new, derived tokens.

The rest of this paper is organized as follows.

\section{Shared Entities using Generics}\label{sec:shared_entities}

\section{An Attack to the Shared Entities Contract}\label{sec:attack}

\section{Fixing the Compilation of the Shared Entities Contract} \label{sec:fix}

\section{Conclusion}\label{sec:conclusion} 

\bibliographystyle{plain}
\bibliography{biblio}

\end{document}
