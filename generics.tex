\documentclass{llncs}

\usepackage{url}
\usepackage{graphicx}
\usepackage[T1]{fontenc}
\usepackage[hidelinks]{hyperref}
\usepackage{pdfpages}
\usepackage{relsize}

\usepackage{svg}
\newcommand{\orcid}[1]{\href{https://orcid.org/#1}{\includesvg[height = 2ex]{svg-inkscape/ORCID_iD}}}

\newcommand{\ie}{\textit{, ie.\ }}

\def\codesize{\smaller}
\def\<#1>{\codeid{#1}}
\newcommand{\codeid}[1]{\ifmmode{\mbox{\codesize\ttfamily{#1}}}\else{\codesize\ttfamily #1}\fi}

\usepackage{listings, xcolor}

\renewcommand{\UrlFont}{\ttfamily\codesize}

\definecolor{verylightgray}{rgb}{.97,.97,.97}

\lstdefinelanguage{Takamaka}{
        keywords=[1]{abstract, break, case, catch, class, continue, default, do
, else, false, finally, for, if, final, implements, extends, import, instanceof, interface, length, new, private, protected, public, return, super, switch, this, throw, true, try, while, var, null}, % generic keywords
        keywordstyle=[1]\color{blue}\bfseries,
        keywords=[2]{boolean, int, long, float, double, byte, short, char, void, enum}, % types; money and time units
        keywordstyle=[2]\color{teal}\bfseries,
        keywords=[3]{@Override,@View,@FromContract,@Payable}, % annotations
        keywordstyle=[3]\color{violet}\bfseries,
        identifierstyle=\color{black},
        sensitive=false,
        comment=[l]{//},
        morecomment=[s]{/*}{*/},
        commentstyle=\color{gray}\ttfamily,
        stringstyle=\color{red}\ttfamily,
        morestring=[b]',
        morestring=[b]"
}

\lstset{
        language=Takamaka,
        backgroundcolor=\color{verylightgray},
        extendedchars=true,
        basicstyle=\scriptsize\ttfamily,
        showstringspaces=false,
        showspaces=false,
        numbers=none,
        numberstyle=\scriptsize,
        numbersep=9pt,
        tabsize=2,
        breaklines=true,
        showtabs=false,
        captionpos=b
}

\begin{document}

\title{Power and Pitfalls of Generic Smart Contracts}
\titlerunning{Power and Pitfalls of Generics for Smart Contracts}
\author{Andrea Benini \and Sara Migliorini \and Fausto Spoto}
\institute{Dipartimento di Informatica, Universit\`a di Verona, Italy}

\maketitle

\begin{abstract}
  TBD
  \keywords{smart contract \and generics \and blockchain}
\end{abstract}

\section{Introduction}\label{sec:introduction}

Blockchains exploit the redundant, concurrent execution of the same
transactions on a decentralized network of many machines
in order to enforce their execution in accordance to
a set of predefined rules. Namely, blockchains make it hard, for a single machine,
to disrupt the semantics of the transactions or their ordering: a misbehaving single machine
gets immediately put out of consensus and isolated. Bitcoin~\cite{Nakamoto08,book-mastering-bitcoin}
has been the first blockchain's success story. Here
transactions are programmed in a non-Turing complete bytecode language,
almost exclusively used to implement transfers of units of coins between \emph{accounts}.

A few years after Bitcoin, another blockchain, called
Ethereum~\cite{Buterin13,AntonopoulosW18}, introduced the possibility of programming
transactions in an actual, imperative and Turing-complete programming language, called Solidity.
Solidity's code is organized in \emph{smart contracts}, that can be seen as
objects that control money.
Ethereum's transactions can hence execute much more than coin transfers. Namely,
they run object constructors and methods, which results in a sort
of \emph{world computer} that persists the same objects in the memory of all the
computers in the blockchain's network.

In Solidity's bytecode,
non-primitive values are referenced through a very general
\<address> type. For instance, a Solidity method
\<child(Person p, uint256 n) returns Person> actually compiles
into \<child(address p, uint256 n) returns address>, losing most
type information~\cite{CrafaPZ19}.
At run time, it is the bytecode that gets executed. Hence,
everything can be passed for \<p>, not just a \<Person>.
The compiler cannot even enforce strong typing
by generating defensive type instance checks and casts, since
values are unboxed in Ethereum: they have no attached
type information at run time,
they are just numerical \emph{addresses}.
It follows that calls to \<Person>'s methods
on \<p> might actually execute arbitrary code, if \<p> is not a \<Person>.
In other words, Solidity is not strongly typed.
Consequently, it is highly discouraged, in Solidity, to call methods on parameters passed
to our methods, such as on \<p> passed to \<child>, since an attacker can pass crafted
objects for \<p>, with arbitrary implementations for their methods,
which can result in the unexpected execution of
dangerous code. This actually happened in the case of the infamous DAO hack~\cite{dao16}, that
costed millions of dollars.

Strong typing is one of the reasons that push towards the adoption
of \emph{traditional} programming languages for smart contracts. For instance,
the Cosmos blockchain~\cite{cosmos} uses Go. The
Hotmoka blockchain~\cite{hotmoka} uses a subset of Java
for smart contracts, called Takamaka~\cite{Spoto19,Spoto20}.
Hyperledger~\cite{hyperldeger} allows Go and Java.
Another reason is the availability of modern
language features, missing in Solidity,
such as \emph{generics}\ie the possibility of using
type variables. Generics are powerful and very useful for programming
smart contracts. In Java source code, they are strongly typed, if no \emph{unchecked operations}
are used~\cite{NaftalinW06}, as it will always be the case in this paper.
However, generics in Java are compiled by \emph{erasure}\ie replaced
by their upwards bound, most often \<Object>. Hence, the risk
is that generics introduce the same kind of attack to the bytecode
as normal reference types do in Solidity.

The contribution of this paper is to show a real-life
use of generics for an actual smart contract used in the support
library of the Takamaka language, and to show that a naive use
of Java generics can lead to a code security vulnerability that
allows an attacker to earn money by exploiting someone else's work.
This paper will provide a fix to that specific issue,
by obliging the compiler to generate defensive checks.
More generally, this paper can be useful for the definition of
bytecode languages for future smart contract languages, by
learning from the weaknesses of Java bytecode.

The rest of this paper is organized as follows.
Sec.~\ref{sec:shared_entities} introduces our real-life Java smart
contracts that uses generics. Sec.~\ref{sec:attack} shows that a naive
deployment of that contract leads to a code vulnerability.
Sec.~\ref{sec:fix} shows a fix to that vulnerability.
Sec.~\ref{sec:conclusion} concludes.

\section{Shared Entities using Generics}\label{sec:shared_entities}

A \emph{shared entity} is something divided into \emph{shares}. Participants,
that hold shares, are called \emph{shareholders} and can dynamically
sell and buy shares. An example of a shared entity is a comporation,
where shares represent units of possess of the company. Another example is
a voting community, where shares represent the voting power of each given voter.
A further example are the validator nodes of a proof of stake blockchain,
where shares represent their voting power and remuneration percentage.

\begin{figure}[t]
  \begin{center}
    \begin{lstlisting}[language=Takamaka]
public interface SharedEntity<S extends PayableContract, O extends Offer<S>>
{
  @View
  BigInteger sharesOf(S shareholder);

  @FromContract(PayableContract.class) @Payable
  void place(BigInteger amount, O offer);

  @FromContract(PayableContract.class) @Payable
  void accept(BigInteger amount, S buyer, O offer);

  class Offer<S extends PayableContract> extends Storage {
    public final S seller;
    public final BigInteger sharesOnSale;
    public final BigInteger cost;
    public final long expiration;

    public Offer
        (S seller, BigInteger sharesOnSale, BigInteger cost, long duration) {

      this.seller = seller;
      this.sharesOnSale = sharesOnSale;
      this.cost = cost;
      this.expiration = now() + duration;
    }
}
    \end{lstlisting}
  \end{center}
  \caption{A simplified subset of our shared entity interface.
  Its full code is available at \url{https://github.com/Hotmoka/hotmoka/blob/master/io-takamaka-code/src/main/java/io/takamaka/code/dao/SharedEntity.java}.}\label{fig:shared_entity}
\end{figure}

In general, two concepts are specific to each implementation of shared entities:
who are the potential shareholders and how offers for selling shares work.
Therefore, we have parameterized the interface of a shared entity with two type variables:
\<S> is the type of the shareholders and \<O> is the type of the offers for selling shares.
Fig.~\ref{fig:shared_entity} shows a simplification of our interface.
It includes an inner class \<Offer> that models sale offers:
it specifies who is the seller of the shares,
how many shares are being sold, the requested price and the expiration of the sale offer.
Implementations might subclass \<Offer> if they need further properties in a sale offer.
By using this class \<Offer>, the \<SharedEntity> interface specifies three methods.
Method \<sharesOf> allows one to know how many shares a potential shareholder (of type \<S>) holds.
It is annotated as \<@View>. In Takamaka, this means that its execution can be performed
\emph{for free}, without paying gas, since it has no side-effects. Method \<place>
activates a new sale offer of shares. This method is annotated as \<@Payable> since
implementations are allowed to require a payment of \<amount> coins for managing the sale.
Method \<accept> is called by who accepts an offer, hence becoming a new shareholder or increasing
its total amount of shares (if it was already a shareholder).
Also this method is \<@Payable>, since its caller must pay the price imposed by the seller
and specified in the \<Offer> instance. Method \<accept> forwards this price to the seller,
which is why shareholders must be \<PayableContract>s: in Takamaka, this means that they
can receive payments.

The annotation \<@FromContract> on both \<place> and \<accept> enforces that only
contracts can call these methods. This is important to attest, explicity, the will
of sharehodlers to sell or buy shares. Therefore,
the callers of both methods should have type \<S>. In Takamaka, this could be written
as \<@FromContract(S.class)>. However, Java does not allow a generic type variable \<S>
in \<S.class>. Because of this limitation,
the best we could write in Fig.~\ref{fig:shared_entity} is \<@FromContract(PayableContract.class)>,
which allows \emph{any} \<PayableContract> to call these methods, not just those of type \<S>.
Since the syntax of the language does not support our abstraction, we will have to
implement explicit dynamic checks in code, as shown in a moment.


\section{An Attack to the Shared Entities Contract}\label{sec:attack}

\section{Fixing the Compilation of the Shared Entities Contract} \label{sec:fix}

\section{Conclusion}\label{sec:conclusion} 

\bibliographystyle{plain}
\bibliography{biblio}

\end{document}
