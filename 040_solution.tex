\section{Fixing the Compilation of the Contract} \label{sec:fix}

The security issue in Sec.~\ref{sec:attack} is due to the
over-permissive erasure of the signature of method \<accept>,
where \<buyer> has type \<PayableContract>.
Therefore, a solution is to oblige the compiler to generate a more
restrictive signature where, in particular, the parameter \<buyer>
has type \<TendermintED25519Validator>: only that type of accounts
must be accepted for the validators.

\begin{figure}[t]
  \begin{center}
    \begin{lstlisting}[language=Java]
public class TendermintValidators
    extends AbstractValidators<TendermintED25519Validator> {

  public TendermintValidators
        (TendermintED25519validator[] validators, BigInteger[] powers) {
    super(validators, powers);
  }

  @Override @FromContract(PayableContract.class) @Payable
  public void accept(BigInteger amount, TendermintED25519Validator buyer, Offer<TendermintED25519Validator> offer) {
    super.accept(amount, buyer, offer);
  }
}
    \end{lstlisting}
  \end{center}
  \caption{The shared entity of the validators of a Hotmoka blockchain built over Tendermint.
  Its full code is available at \url{https://github.com/Hotmoka/hotmoka/blob/master/io-takamaka-code/src/main/java/io/takamaka/code/governance/tendermint/TendermintValidators.java}.}\label{fig:solution}
\end{figure}

The fixed code is shown in Fig.~\ref{fig:solution}. The only difference is that method
\<accept> has been redefined to force the correct type for \<buyer>. For the rest, that method
delegates to its implementation in \<AbstractValidators>.
It is important to investigate which is the Java bytecode generated from
the code in Fig.~\ref{fig:solution}. Since Java bytecode does not allow one to redefine a method
and modify its argument types, the compiled bytecode actually contains \emph{two}
\<accept> methods, as follows:

\begin{lstlisting}[language=JavaBytecode]
public class TendermintValidators extends AbstractValidators {
  ...
  
  public void accept(BigInteger,TendermintED25519Validator,Offer)
   aload_0
   aload_1
   aload_2
   aload_3
   invokespecial AbstractValidators.accept(BigInteger,PayableContract,Offer)
   return

  // synthetic bridge method
  public void accept(BigInteger,PayableContract,Offer)
   aload_0
   aload_1
   aload_2
   checkcast TendermintED25519Validator
   aload_3
   invokevirtual accept(BigInteger,TendermintED25519Validator,Offer)
   return
}
\end{lstlisting}

\noindent
The first \<accept> method is the compilation of that from Fig.~\ref{fig:solution}:
it delegates to the \<accept> method of the superclass \<AbstractValidators>. The second
\<accept> method is a \emph{bridge method} that the compiler generates in order to guarantee
that all calls to the erased signature \<accept(BigInteger,PayableContract,Offer)> actually
get forwarded to the first, redefined \<accept>. It casts its \<buyer> argument
into \<TendermintED25519Validator> and calls the first \<accept>. This
bridge method and its checked cast guarantee that only \<TendermintED25519Validator>s
can become validators. For instance, an instance of \<Attacker> (Fig.~\ref{fig:attacker})
cannot be passed to the first \<accept> (type mismatch) and makes the second \<accept>
fail with a class cast exception. The \emph{Consistency of Shareholders} holds
for instances of \<TendermintValidators> now and the attack in Sec.~\ref{sec:attack} cannot occur
anymore.

\todo{Si potrebbe dire che la soluzione e' simile ad una gestione dei generics eterogenea? che pero' deve essere fatta a mano ridefinendo i metodi, invece che essere generata in automatico dal compilatore? }

