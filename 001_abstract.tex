Generics are a powerful feature of programming languages that allows one
to write highly reusable code.
%
More specifically, they are based on the use of type placeholders in order
to produce parametrized code, that can be instantiated for each
concrete type provided for them. 
%
In many programming languages, such as Java, they are implemented by
\emph{erasure}\ie replaced by their upper bound type during compilation into bytecode.
%
This paper investigates the use of generics in the
context of smart contracts for blockchain, in order to implement
a contract for \emph{shared entities} (such as a company shared
by its shareholders),
by using the Hotmoka blockchain whose contracts are written in Java.
%
The considered case study is particularly important since
the validators' set of the blockchain itself is
a special case of shared entities.
The analysis shows that the power of generics comes at the risk of a too permissive
typing of the compiled code, due to the erasure mechanism, with a consequent possible attack
to the validators' set. This paper proposes a solution
that forces the compiler to generate more precise type information than
those arising by erasure.
