\section{Conclusion}\label{sec:conclusion} 

This paper has shown that generic types can be useful in the definition
of smart contracts, but introduce risks of security since in many
programming languages, including Java, they get erased at compile time
into types that might be too permissive for low-level calls, such as
those that start blockchain transactions. Note that using a programming
language without generics is not the solution: Solidity has no generics
and consequently erases \emph{all} reference types into \<address>.
That is the worst possible erasure.

The solution in this paper has been to redefine a method with an argument
of generic type in such a way to call its superclass
(see \<accept> in Fig.~\ref{fig:solution}). This fixes the security risk,
but cannot be regarded as an elegant solution. It is just a trick that
works because it forces the compiler to generate some kind of bytecode.
A \emph{smarter} compiler might recognize the redefined \<accept> as
\emph{useless} and just remove it. This would recreate the issue that
we have just solved. That is, our solution works only for the way
compilers compile \emph{today}.

With hindsight, it is questionable to have implemented generics by erasure
and code instrumentation (bridge methods). If generics would be present
and checked at bytecode level, the attack in Sec.~\ref{sec:attack} would just
be impossible. Currently, generics can only exist as bytecode annotations
that are not mandatory and are ignored by the Java virtual machine
that runs the bytecode. The same consideration might be applied beyond
generics: many features of modern programming
languages have no direct low-level
counterpart but are implemented via instrumentation, from
inner classes to closures. This is fine at source level, but allows
low-level calls to easily circumvent
the encapsulation guarantees of the language. When embedded in a permissionless
blockchain, such features become dangerous attack surfaces.
