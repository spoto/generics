Smart contracts are pieces of code that are deployed and executed in the context of a blockchain infrastructure in order automatically enforce some effects when particularly events occur. The writing of smart contracts is a complex and critical activity which can benefit from the use of high-level features of programming languages, like generics. In many programming languages, such as Java, generics are implemented by \emph{erasure}\ie replaced by their upper bound type during compilation into bytecode. This is safe at source level, since the compiler takes care of checking that types are correct, before erasure. However, the types of the generated bytecode are erased and consequently weaker. In a permissionless blockchain, where every user can call the bytecode of smart contracts installed by other users, these weaker types pose a risk of attack. In this paper we start from a real security issue that we found while using generics for writing smart contracts that implement \emph{shared entities} for the Hotmoka blockchain, and we propose a possible solution based on an appropriate code rewriting.
