Smart contracts are pieces of code that are deployed and executed in the context of a blockchain infrastructure in order to automatically enforce some effects when particular events occur. The writing of smart contracts is a complex and critical activity that can benefit from the use of high-level features of programming languages, like generics. In many programming languages, such as Java, generics are implemented by \emph{erasure}\ie replaced by their upper bound type during compilation into bytecode. This is safe at source level, since the compiler takes care of checking that types are correct, before erasure. However, the types of the generated bytecode are erased and consequently weaker. In a permissionless blockchain, where every user can call the bytecode of smart contracts installed by other users, these weaker types pose a risk of attack. This paper focuses on a real security issue found while using generics for writing smart contracts that implement \emph{shared entities} for the Hotmoka blockchain, that can lead to stealing the remuneration of validator nodes, and proposes a patch based on appropriate code rewriting.
