Generics are a powerful feature of programming languages that allows one
to write highly reusable code.
%
More specifically, they are based on the use of type placeholders in order
to produce parametrized code, that can be instantiated for each
concrete type provided for the placeholders.
%
In many programming languages, such as Java, generics are implemented by
\emph{erasure}\ie replaced by their upper bound type during compilation into bytecode.
This is safe at source level, since the compiler takes care of checking that
types are correct, before erasure. However, the types of the generated bytecode
are erased and consequently weaker. In a permissionless blockchain, where
every user can call the bytecode of smart contracts installed by other users,
these weaker types pose a risk of attack.
%
Namely, this paper originated from a real security issue that we found
while using generics for writing
smart contracts that implement
\emph{shared entities} (such as a company shared by its shareholders),
for the Hotmoka blockchain, in a subset of Java called Takamaka.
%
The considered case study is particularly important since
the validators' set of the blockchain itself is
a special case of shared entity.
The analysis shows that the power of generics comes at the risk of
allowing the intrusion of
a validator of an illegal type, that exploits the work of another
validator in order to earn validation rewards, an actual attack
to the validators' set. This paper proposes a solution
that forces the compiler to generate more precise type information than
that generated by erasure.
