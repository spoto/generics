%
\documentclass[journal,onecolumn, 11pt]{IEEEtran}

\usepackage{times,epsf}
\usepackage{graphicx}
\usepackage{color}
\usepackage{url}


\usepackage{amsfonts}
\usepackage{amsmath,amssymb}
\usepackage{xcolor}


\setlength{\textwidth}{5.5in}
\setlength{\hoffset}{0.4in}

\setlength{\oddsidemargin}{0.0in}
\setlength{\evensidemargin}{0.0in}

\newcommand{\todo}[1]{\textcolor{red}{#1}}

%\makeatletter

% doublespace
% \def\baselinestretch{0.93}
%\def\baselinestretch{1.4}

% more than .95 of text and figures

\def\topfraction{.95}
\def\floatpagefraction{.95}
\def\textfraction{.05}

\newcommand{\BOX}[1]
{
  {\it
%    \medskip
    \begin{center}
      \begin{tabular}{|c|}
        \hline
        \parbox{0.97\columnwidth}{
          \medskip
          #1
          \medskip} \\
        \hline
      \end{tabular}
    \end{center}
%    \medskip
  }
}


\begin{document}

\title{{\LARGE On the Use of Generic Types for Smart Contracts}}

\author{{\normalsize F. Spoto, S. Migliorini, M. Gambini, A. Benini}}

\date{}

\maketitle

Dear Editors,

We are pleased to submit for your consideration our paper: ``On the Use of Generic Types for Smart Contracts''. This work addresses the benefits and weaknesses in using high-level programming featuare, like generics, in the development of blockchain-based smart contracts. 


This work is an extended version of our previous paper ``Power and Pitfalls of Generic Smart Contracts'' published in the third IEEE International Conference on Blockchain Computing and Applications. In comparison to that paper, Sec. 3 and Sec. 5 are new; while all other sections have been expanded with several additional details and enriched with many explanatory figures.

\hspace{7cm}

Kinds regards.

The authors

\end{document}


