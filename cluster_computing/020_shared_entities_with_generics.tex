\section{Shared Entities using Generics}\label{sec:shared_entities}

A \emph{shared entity} is concept that often arises in blockchain
applications. Namely, a shared entity is something divided into \emph{shares}. Participants,
that hold shares, are called \emph{shareholders} and can dynamically
sell and buy shares. An example of a shared entity is a corporation,
where shares represent units of possess of the company. Another example is
a voting community, where shares represent the voting power of each given voter.
A further example are the validator nodes of a proof of stake blockchain,
where shares represent their voting power and remuneration percentage.

\begin{figure}[ht]
  \begin{center}
    \begin{lstlisting}[language=Takamaka]
public interface SharedEntity<S extends PayableContract,
                              O extends Offer<S>> {

  @View
  BigInteger sharesOf(S shareholder);

  @FromContract(PayableContract.class) @Payable
  void place(BigInteger amount, O offer);

  @FromContract(PayableContract.class) @Payable
  void accept(BigInteger amount, S buyer, O offer);

  List<S> getShareholders();

  class Offer<S extends PayableContract> extends Storage {
    public final S seller;
    public final BigInteger sharesOnSale;
    public final BigInteger cost;
    public final long expiration;

    public Offer
        (S seller, BigInteger sharesOnSale, 
         BigInteger cost, long duration) {

      this.seller = seller;
      this.sharesOnSale = sharesOnSale;
      this.cost = cost;
      this.expiration = now() + duration;
    }

    @View
    public boolean isOngoing() {
      return now() <= expiration;
    }
  }
}
    \end{lstlisting}
  \end{center}
  \caption{A simplified part of our shared entity interface.
  Its full code is available at \textsf{\url{https://github.com/Hotmoka/hotmoka/blob/master/io-takamaka-code/src/main/java/io/takamaka/code/dao/SharedEntity.java}}.}\label{fig:shared_entity}
\end{figure}

In general, two concepts are specific to each implementation of shared entities:
who are the potential shareholders and how offers for selling shares work.
Therefore, one can parameterize the interface of a shared entity with two type variables:
\<S> is the type of the shareholders and \<O> is the type of the offers for selling shares.

We are developing for the Hotmoka blockchain, that uses smart contracts written in a subset
of Java called Takamaka.
Fig.~\ref{fig:shared_entity} shows a simplification of our interface for shared entities.
It includes an inner class \<Offer> that models sale offers:
it specifies who is the seller of the shares,
how many shares are being sold, the requested price and the expiration of the offer.
Method \<isOngoing> checks if an offer has not expired yet.
Implementations can subclass \<Offer> if they need more specific offers.
By using class \<Offer>, the \<SharedEntity> interface specifies four methods.
Method \<sharesOf> allows one to know how many shares a potential \<shareholder> (of type \<S>) holds.
It is annotated as \<@View>. In Takamaka, this means that its execution can be performed
\emph{for free}, without paying gas, since it has no side-effects. Who wants to sell
(some of) its shares calls method \<place> with a sale \<offer>.
This method is annotated as \<@Payable> since
implementations are allowed to require a payment of \<amount> coins for managing the sale.
Who buys the shares calls method \<accept> with the accepted \<offer>
and with itself as \<buyer> (the reason will be explained soon)
and becomes a new shareholder or increases
its cumulative number of shares (if it was a shareholder already).
Also this method is \<@Payable>, since its caller must pay \<amount> $\ge$ \<offer.cost>
coins to the seller.
This means that shareholders must be able to receive payments and that
is why \<S extends PayableContract>: the \<PayableContract>s, in Takamaka, are those that can receive
payments.
Method \<getShareholders> yields the list of the current shareholders of the entity.
It is not \<@View>, since it creates a new list, which is a side-effect.

The annotation \<@FromContract> on both \<place> and \<accept> enforces that only
contracts can call these methods.
These callers must be (old or new) shareholders,
hence they must have type \<S>. In Takamaka, this could be written
as \<@FromContract(S.class)>. Unfortunately, Java does not allow a generic type variable \<S>
in \<S.class>. Due to this syntactical limitation of Java,
the best we could write in Fig.~\ref{fig:shared_entity} is \<@FromContract(PayableContract.class)>,
which allows \emph{any} \<PayableContract> to call these methods, not just those of type \<S>.
Since the syntax of the language does not support our abstraction, we will have to
program explicit dynamic checks in code, as shown later, and this will be the reason of the
parameter \<buyer> in \<accept>.

Let us state an important property about shared entities:

%\begin{tcolorbox}
  \begin{center}\emph{Consistency of Shareholders}\end{center}
  If \<se> is a {\codesize\texttt{SharedEntity<S,O>}}
  then the elements contained in the list \<se.getShareholders()> have type \<S>.
%\end{tcolorbox}

\noindent
This property is important since it states that we can trust the type \<S> of
the shareholders: if we create a \<SharedEntity> and fix a specific type \<S>
for its shareholders, then only instances of \<S> will actually manage to become shareholders.

\begin{figure}[t]
  \begin{center}
    \begin{lstlisting}[language=Takamaka]
public class SimpleSharedEntity
      <S extends PayableContract,O extends Offer<S>>
      extends Contract
      implements SharedEntity<S,O> {

  private final StorageTreeMap<S,BigInteger> shares = 
     new StorageTreeMap<>();
  private final StorageSet<O> offers = 
     new StorageTreeSet<>();        

  public SimpleSharedEntity
    (S[] shareholders, BigInteger[] shares) {
    require(shareholders.length == shares.length, 
            "length mismatch");
    for (int pos = 0; pos < shareholders.length; pos++)
      addShares(shareholders[pos], shares[pos]);
  }

  @Override @View
  public final BigInteger sharesOf(S shareholder) {
    return shares.getOrDefault
      (shareholder, BigInteger.ZERO);
  }

  @Override @FromContract(PayableContract.class) @Payable
  public void place(BigInteger amount, O offer) {
    require(offer.seller == caller(), 
            "not authorized to sell");
    require(shares.containsKey(offer.seller), 
            "only shareholders can sell");
    require(sharesOf(offer.seller)
        .subtract(sharesOnSaleOf(offer.seller))
        .compareTo(offer.sharesOnSale) >= 0, 
        "not enough shares to sell");
    offers.add(offer);
  }

  @Override @FromContract(PayableContract.class) @Payable
  public void accept(BigInteger amount, S buyer, O offer) {
    require(caller() == buyer, 
            "only the future owner can buy the shares");
    require(offers.contains(offer), "unknown offer");
    require(offer.isOngoing(), 
            "the sale offer is not ongoing anymore");
    require(offer.cost.compareTo(amount) <= 0, 
            "not enough money");
    offers.remove(offer);
    removeShares(offer.seller, offer.sharesOnSale);
    addShares(buyer, offer.sharesOnSale);
    offer.seller.receive(offer.cost);
  }

  private BigInteger sharesOnSaleOf(S shareholder) {
    return offers.stream()
      .filter(offer -> offer.seller == shareholder && 
              offer.isOngoing())
      .map(offer -> offer.sharesOnSale)
      .reduce(ZERO, BigInteger::add);
  }

  @Override
  public List<S> getShareholders() {
    return shares.keyList();
  }
}
    \end{lstlisting}
  \end{center}
  \caption{A simplified part of our implementation of the shared entity interface.
  Its full code is available at \textsf{\url{https://github.com/Hotmoka/hotmoka/blob/master/io-takamaka-code/src/main/java/io/takamaka/code/dao/SimpleSharedEntity.java}}.}\label{fig:simple_shared_entity}
\end{figure}

Fig.~\ref{fig:simple_shared_entity} shows our implementation of the \<SharedEntity> interface
in Fig.~\ref{fig:shared_entity}
by using two fields: \<shares> maps each shareholder to its amount of shares and
\<offers> collects the offers that have been placed.
The constructor initially populates the map \<shares>.
Method \<sharesOf> simply accesses
\<shares>, by using zero as default. Method \<place> requires its \<caller()> to be
the seller identified in the \<offer>. This forbids shareholders to sell shares on behalf of others.
Moreover, this guarantees that the caller has type \<S>, like \<offer.seller>.
As we said before, this cannot be expressed with the syntax of the language.
Method \<place> further requires the seller to be a shareholder with at least \<offer.sharesOnSale>
shares not yet placed on sale. This forbids to oversell more shares
than one owns. At the end, \<place> adds the \<offer> to the set of \<offers>.
Method \<accept> requires that who calls the method must be \<buyer>. Hence, successful
calls to \<accept> can only pass the same caller for \<buyer>. This is a trick to enforce the
caller to have type \<S>, since the syntax of the language does not allow one to express it,
as we explained before. Then \<accept> requires the \<offer> to exist, to be still ongoing
and to cost no more than the \<amount> of money provided to \<accept>. If that is the case,
the \<offer> is removed from the \<offers>, shares are moved from seller to buyer (code not
shown in Fig.~\ref{fig:simple_shared_entity}) and the seller of the \<offer>
receives the required price \<offer.cost>.
Method \<getShareholders()> yields the list of the keys in the domain of map \<shares>.

It turns out that the \emph{Consistency of Shareholders} property holds if Java source code
creates and populates \<SimpleSharedEntity>s.
Namely, the code in Fig.~\ref{fig:simple_shared_entity}
does not use unchecked casts, hence it is strongly-typed~\cite{NaftalinW06} and
the map \<shares> actually holds values of type \<S> in its domain, only.
For this consistency result, we had to pay the price
of the dummy \<buyer> argument for method \<accept>. Without that argument,
\emph{Consistency of Shareholders} would not hold, since we could only write
\<addShares((S) caller(), offer.sharesOnSale)> in the implementation of \<accept> in
Fig.~\ref{fig:simple_shared_entity}, with an unchecked cast that makes the code
non-strongly-typed. In that case, also contracts not of type \<S> could call \<accept>
and become shareholders.

There is, however, a problem with the reasoning
in the previous paragraph. Namely, absence of unchecked
operations guarantees strong typing of Java \emph{source} code. But what is installed
and executed in blockchain is the Java bytecode that has been derived from
the compilation of the code in Fig.~\ref{fig:simple_shared_entity}.
Malicious users might install in blockchain some manually crafted bytecode,
not derived from its Java source code compiled together with the source code
in Fig.~\ref{fig:simple_shared_entity}.
That crafted code might
call the methods of \<SimpleSharedEntity>s in order to attack
that contract. In particular,
the signature of method \<accept> declares a parameter \<buyer> of type \<S> at source code level, but
its compilation into Java bytecode declares an erased
parameter \<buyer> of type \<PayableContract> instead.
It follows that an attacker can install in blockchain a snippet of bytecode that calls
\<accept> and passes \emph{any} \<PayableContract>, not only those that are instances of \<S>:
the \emph{Consistency of Shareholders} property is easily violated at bytecode level.

In general, the problem arises whenever a method declares a formal parameter of generic type,
such as \<buyer> in method \<accept> of Fig.~\ref{fig:shared_entity}. Next section shows
the actual significance of the consequent security risk.
