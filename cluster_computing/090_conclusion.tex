\section{Conclusion}\label{sec:conclusion} 

This paper has shown that generics are useful in the definition
of smart contracts and can simplify the development of rather complex
code such as that for shared entities, and support code reuse, for
instance to implement the validators set of a blockchain network.
However, this paper has shown that generic types
introduce risks of security as well.
Namely, many
programming languages, including Java, erase them at compile time
into types that might be too permissive for low-level calls, such as
those that are started by blockchain transactions. Note that the use of a programming
language without generics is not the solution: Solidity has no generics
and consequently erases \emph{all} reference types into \<address>.
That is the worst possible erasure.

The solution in this paper has been to redefine the methods that have an argument
of generic type, in such a way to call their superclass
(see the case of \<accept> in Fig.~\ref{fig:solution}). This fixes the security risk,
but cannot be regarded as the definite solution to the problem. It is just a trick that
works because it forces the compiler to generate some specific kind of bytecode.
A \emph{smarter} compiler might recognize the redefined \<accept> as
\emph{useless} and just remove it. This would recreate the issue that
has been just solved. That is, the solution in this paper works only for the way
compilers compile \emph{today}.

With hindsight, it is questionable to have implemented generics by erasure
and code instrumentation (bridge methods). If generics would be present
and checked at bytecode level, the attack in Sec.~\ref{sec:attack} would just
be impossible. Currently, generics can only exist as bytecode annotations
that are not mandatory and are ignored by the Java virtual machine
that runs the bytecode. The same consideration might be applied beyond
generics: many features of modern programming
languages have no direct low-level
counterpart but are implemented via instrumentation. Examples are
inner classes and closures (lambda expressions). This is fine at source level, but allows
low-level calls to easily circumvent
the encapsulation guarantees of the language. When embedded in a permissionless
blockchain, such features become dangerous attack surfaces. This paper has shown
the attack surface due to redefinition of methods with a generic parameter.
But another example is the use of instrumented methods to allow access to
private state from inner classes: since inner classes are compiled into distinct
bytecode classes, the compiler adds non-private accessors to the
private state. These accessors cannot be used
at source level, but can be called at bytecode level to gain access to private state.
This paper does not provide a solution to this other issue, but this further example
makes it clear that the attack surface is larger than what described here.
